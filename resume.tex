%-------------------------
% Resume in Latex
% Resume Author : Arthur Hertz
% Template Author : Sourabh Bajaj
% Derived from : https://github.com/sb2nov/resume
% License : MIT
%------------------------
 
\documentclass[letterpaper,11pt]{article}

\usepackage{latexsym}
\usepackage[empty]{fullpage}
\usepackage{titlesec}
\usepackage{marvosym}
\usepackage[usenames,dvipsnames]{color}
\usepackage{verbatim}
\usepackage{enumitem}
\usepackage[pdftex]{hyperref}
\usepackage{fancyhdr}


\pagestyle{fancy}
\fancyhf{} % clear all header and footer fields
\fancyfoot{}
\renewcommand{\headrulewidth}{0pt}
\renewcommand{\footrulewidth}{0pt}

% Adjust margins
\addtolength{\oddsidemargin}{-0.375in}
\addtolength{\evensidemargin}{-0.375in}
\addtolength{\textwidth}{1in}
\addtolength{\topmargin}{-.5in}
\addtolength{\textheight}{1.0in}

\urlstyle{same}

\raggedbottom
\raggedright
\setlength{\tabcolsep}{0in}

% Sections formatting
\titleformat{\section}{
  \vspace{-4pt}\scshape\raggedright\large
}{}{0em}{}[\color{black}\titlerule \vspace{-5pt}]

%-------------------------
% Custom commands
\newcommand{\resumeItem}[2]{
  \item\small{
    \textbf{#1}{: #2 \vspace{-2pt}}
  }
}

\newcommand{\resumeSubheading}[4]{
  \vspace{-1pt}\item
    \begin{tabular*}{0.97\textwidth}{l@{\extracolsep{\fill}}r}
      \textbf{#1} & #2 \\
      \textit{\small#3} & \textit{\small #4} \\
    \end{tabular*}\vspace{-5pt}
}

\newcommand{\resumeSubItem}[2]{\resumeItem{#1}{#2}\vspace{-4pt}}

\renewcommand{\labelitemii}{$\circ$}

\newcommand{\resumeSubHeadingListStart}{\begin{itemize}[leftmargin=*]}
\newcommand{\resumeSubHeadingListEnd}{\end{itemize}}
\newcommand{\resumeItemListStart}{\begin{itemize}}
\newcommand{\resumeItemListEnd}{\end{itemize}\vspace{-5pt}}

%-------------------------------------------
%%%%%%  RESUME STARTS HERE  %%%%%%%%%%%%%%%%%%%%%%%%%%%%


\begin{document}

%----------HEADING-----------------
\begin{tabular*}{\textwidth}{l@{\extracolsep{\fill}}r}
  \textbf{\href{}
               {\Large Arthur Hertz}} &
                \begin{tabular}{@{\extracolsep{10pt}}rr}
                    Email    : & \href{mailto:arthertz@iu.edu}{arthertz@iu.edu} \\
                    LinkedIn : & \href{https://www.linkedin.com/in/arthertz/}{linkedin.com/in/arthertz/} \\
                    GitHub   : & \href{https://github.com/arthertz}{github.com/arthertz} \\
                    Mobile   : & (812) 606-3357 \par \\
                \end{tabular}
\end{tabular*}


%-----------EDUCATION-----------------
\section{Education}
  \resumeSubHeadingListStart
    \resumeSubheading
      {Indiana University Bloomington}{Bloomington, IN}
      {Bachelor of Science in Computer Science;  GPA: 3.5}{Aug. 2019 -- May. 2023}
    \resumeSubheading
        {Indiana University Bloomington}{Bloomington, IN}
        {Bachelor of Science in Mathematics;  GPA: 3.5;}{Aug. 2019 -- May. 2023}
        \\ \vspace{10pt} \textit{\small Selected Coursework: Data Structures, Probability Theory, Algorithm Design and Analysis, x86 Compiler Construction,\\Computer Systems (C and Unix), Programming Language Theory, Modal Logic}
  \resumeSubHeadingListEnd

%--------PROGRAMMING SKILLS------------
\section{Programming Skills}
 \resumeSubHeadingListStart
   \item{
     \textbf{Languages}{: Python, Racket, C/C++, Java, x86 assembly,  C\#, Javascript}}
 \resumeSubHeadingListEnd

%-----------EXPERIENCE-----------------
\section{Experience}
  \resumeSubHeadingListStart

    \resumeSubheading
      {Indiana University Bloomington}{Bloomington, IN}
      {Undergraduate Instructor}{Aug 2020 - Present}
      \resumeItemListStart
        \resumeItem{Programming Language Theory (CSCI-C311)}
          {Teach the design and implementation of interpreters with features like lazy evaluation, dynamic and static type systems, and transpiling programs from Scheme to C. Grade student work, write feedback and hold regular office hours. Taught Fall 2021.}
        \resumeItem{Introductory Computer Science (CSCI-H211)}
          {Teach introductory programming at the honors level, with topics including higher order functions, recursion, type systems, and good design habits. Grade student work, write feedback, hold regular office hours. Lead Friday laboratories for the honors students. Taught Fall 2020.}
      \resumeItemListEnd

    \resumeSubheading
      {Indiana University Geometry Lab}{Bloomington, Indiana}
      {Undergraduate Researcher}{Jan 2020 - May 2020}
      \resumeItemListStart
        \resumeItem{Research}
          {Worked with graduate students to study hyperbolic 3-manifolds and compute their normalized volume. Presented to an audience of professors and graduate students on current research relating to hyperbolic metrics, Mobius transformations, theory of 3-manifolds.}
      \resumeItemListEnd

    \resumeSubheading
      {X-Force Fellowship}{Bloomington, Indiana}
      {Software Engineer}{Summer 2020}
      \resumeItemListStart
        \resumeItem{Architecture}
          {Designed and simulated agents capable of mapping unknown spaces and avoiding obstacles.}
        \resumeItem{Optimization}
          {Used algorithmic complexity knowledge to improve simulation performance and evaluate algorithms for cooperative pathfinding.}
        \resumeItem{Visualization}
          {Created software visualizations for the simulation and agent performance. Presented findings and data internally through weekly milestones and blog posts, culminating in a technical report.}
      \resumeItemListEnd

  \resumeSubHeadingListEnd


%-----------PROJECTS-----------------
\section{Projects}
  \resumeSubHeadingListStart
    \resumeSubItem{Racketscript (current project)}
      {Worked to add tail call recursion and first-class continuations to Racketscript, an open source project to compile from Racketscript into Javascript.}
    \resumeSubItem{Incremental Racket Compiler (current project)}
      {Developed a compiler from a subset of Racket to x86-64 assembly using the Nanopass Framework for compiler construction, with unit tests for each pass and integration tests for the generated assembly. Current features include 64-bit arithmetic, register allocation with Brelaz graph coloring, move biasing, and partial evaluation.}
    \resumeSubItem{Procedural Terrain Project}
      {Developed a realtime game with custom procedural terrain written in C\#. Used spacial noise algorithms to generate interesting random terrain. Wrote a custom mesh generation algorithm and custome chunk management algorithms to facilitate realtime performance.}
    \resumeSubItem{Autonomous Navigation Project}
      {Developed a priority-based realtime collaborative pathfinding and avoidance system for a multi-agent environment. A network of agents used a modified A* algorithm to map out a simulated unknown environment while avoiding obstacles. Implemented in Python and funded by the X-Force Fellowship.}
    \resumeSubItem{Python\_RT}
      {A small multithreaded raytracer written in python. Simulates the transport of light on diffuse materials using Monte-Carlo integration.}
  \resumeSubHeadingListEnd
%-------------------------------------------
\end{document}